\documentclass[french,12pt]{article}
\usepackage[T1]{fontenc}
\usepackage[utf8]{inputenc}
\usepackage{kpfonts}
\usepackage{babel}
\usepackage{microtype}
\usepackage{graphicx}
\usepackage{geometry}
\usepackage{hyperref}
\usepackage[backend=biber]{biblatex}
\bibliography{document.bib} % or
\geometry{hmargin=2.5cm,vmargin=1.5cm}

\title{État de l'art - spatial}
\begin{document}
\maketitle

\section{Pratiques}
Art : \cite{lodi_spatial_2014}
Problématiques courantes : reconnaissance, etc. Ici nous nous intéressons à l'écriture.
\subsection{Pratique appliquée à l'audio}
- est utile pour l'audio en réalité augmentée\cite{lemordant_augmented_2010}
\subsubsection{Composition et écriture}
\cite{delle_monache_sonic_2011}
\cite{colafrancesco_bibliotheque_2013}
\cite{fober_les_2015}
\subsubsection{Positionnement spatial}
\cite{wozniewski_framework_2006}
\subsubsection{Restitution spatiale}
- Rendu de son dans des espaces virtuels \cite{funkhouser_beam_1998,cheng_design_2014,tsingos_fast_1998,taylor_guided_2012,raghuvanshi_parametric_2014,rodriguez_performance_2014}
Outils pour la création de tels espaces (funkhouser), 
- Méthodes : WFS, Amibisonie... \cite{lim_3d_2015}
\cite{perez-lopez_3dj_2015}
\cite{noisternig_3d_2003}\cite{bates_comparative_2007},\cite{tang_assistive_2014},\cite{sasamoto_controlling_2013},\cite{delerue_spatialisation_2004}

- Utilisation de données spatiales pour la création artistique \cite{naveda_topos_2014}

- Les objets qui étaient auparavant simplement connectés ont maintenant des infos de positions (téléphone, etc.) avec lesquelles il est possible de travailler (\cite{beal_spacetime_2015})
\subsection{Pratique liée à l'interaction}
\subsubsection{Jeu vidéo}
\cite{le_prado_ecriture_2013}
\cite{salazar_modelisation_2004}

\subsubsection{Systèmes interactifs}
\cite{chalon_realite_2004}
\cite{jankowski_advances_2015}
\cite{shen_blowbrush:_2014}
\cite{gustafson_imaginary_2010}
\subsubsection{Muséographie}
\cite{adhitya_composing_2012}
\cite{michael_comparative_2010}
\cite{azough_modeet_2014}
\cite{kidd_multi-touch_2011}
\subsubsection{Robots}
\cite{lee_virtual_2014} % NOTE : voir dans les références 1er paragraphe. 

\section{Manipulation et écriture}

\subsection{Extraction de données}
\cite{li_aesthetic_2009} (cf. slide 26)
\subsection{Trajectoires et animation}
\cite{santosa_direct_2013}
\cite{kazi_kitty:_2014,scott_physink:_2013}

\cite{garcia_jeremie_processing_2015,garcia_towards_2015}
\cite{wakefield_cosm:_2011}
\cite{wagner_introducing_2014}
\cite{melchior_authoring_2005} 
\cite{bresson_spatial_2012}
\cite{wozniewski_spatosc:_2012}
\cite{favory_trajectoires:_2015}
\cite{casas_4d_2013}


\subsubsection{Manipulation virtuelle}
\cite{jacob_design_2014} % NOTE: état de l'art intéressant au début.
\cite{cohen_interface_1999}
\subsubsection{Manipulation réelle}
\cite{grossman_interface_2003}


\subsection{Autres approches}
\cite{andriamarozakaniaina_du_2012} : Génération d'espace à partir de texte !
\cite{van_nort_lom_2006}
 - Zones ? logiciels de 3d paramétrique, etc.

\section{Modélisation}
\cite{porter_handbook_2008}
\cite{grenon_formal_2003}
\cite{zhang_timed_2014}
\cite{benford_spatial_1993}

\cite{matlage_every_2011}
\subsection{Modèles de données}
- OWL : \cite{mefteh_approche_2013}

- Adaptés à l'audio
. SpatDIF et compagnie \cite{peters_spatial_2013}\cite{kendall_towards_2008}
. \cite{kondoz_object-based_2014}
- Généralistes
\cite{hudak_arrows_2003}

Approches langage : (pour reconnaissance : \cite{spranger_recruitment_2011}\cite{spranger_emergent_2012}, )

- Modèles qualitatifs : \cite{chen_survey_2015}, \cite{bhatt_geospatial_2014}, \cite{schlieder_qualitative_1996,dorr_qualitative_2014}. Spatio-temporel : \cite{hazarika_qualitative_2012}. \cite{clementini_global_1997}.

- Voxels : \cite{kaufman_volume_1993}
- Quadtrees : \cite{eppstein_skip_2008}
\subsection{Problèmes géométriques et calculatoires}

\section{Rendu et visualisation}
\cite{hortner_spaxels_2012}
\subsection{Sur écran}
\subsection{Autre}
\printbibliography
\end{document}