\documentclass[french,12pt]{article}
\usepackage[T1]{fontenc}
\usepackage[utf8]{inputenc}
\usepackage{kpfonts}
\usepackage{babel}
\usepackage{microtype}
\usepackage{graphicx}
\usepackage{geometry}
\usepackage{hyperref}
\usepackage{acronym}
\usepackage[backend=biber]{biblatex}
\bibliography{document.bib} % or
\geometry{hmargin=2.5cm,vmargin=1.5cm}

\title{État de l'art - spatial}

\acrodef{ENP}{Expressive Notation Package}
\acrodef{GPU}{Graphics Processing Unit}
\begin{document}
\maketitle

\section{Introduction}
Ce document vise à présenter un état de l'art des partiques impliquant la manipulation de 
données spatiales dans un cadre artistique, interactif et muséographique.

Nous présentons notamment les besoins des concepteurs, ainsi que les diverses solutions qui 
peuvent y être apportées, que ce soit par la conception de logiciels dédiés ou bien 
par des modélisations formelles des notions d'espace.
 
 
 - Délimitation du périmètre
 - Définitions de l'espace dans les différentes pratiques ?
  * Lien très fort avec la VR et les mondes virtuels - jeux vidéos.
  * Audio spatialisé
  * Espace de données et manipulation de données multi-dimensionnelles
  * Muséographie
  * Big data ?
  * Reconaissance d'images 
  * Conception de spectacles / chorégraphie / etc

Beaucoup d'overlap.
  
Le problème qui nous intéresse principalement est celui de l'écriture. Cela implique des recherches sur les applications faisant usage d'écriture spatiale, des logiciels interagissant avec des données spatiales, des modèles aptes à représenter de telles données et des paradigmes d'interfaces utilisateur permettant de les modifier et de les retravailler. 



\section{Pratiques}
Art : \cite{lodi_spatial_2014}. L'espace dans une dimension collective. Exemple : Steve Eley, Licorne Rose et Invisible. Structure virtuelle apposée dans le monde réel et visible seulement à l'aide d'une application dédiée. Progression de la réalité virtuelle : la tendance est dans l'augmentation de la réalité par l'ajout d'objets qui extraient ou rendent des données (Manovich, 2005). "Spatial Art overlays and unites several spaces into one, making artistic use of time, movement and data or information in a space defined by growth in technological interaction, i.e., a data space."

Nous étudierons d'abord le cas de l'audio et de la spatialisation en audio (mettre historique).

\subsection{Pratique appliquée à l'audio}
Utilisation dans la musique, ainsi que dans des applications de réalité augmentée.

Problème principal dans cadre musical : couplage fort entre l'écriture et le moyen de restitution. Par exemple il est souvent nécessaire d'ajuster les paramètres de spatialisation quand on change l'endroit ou se déroule une représentation. %TODO trouver source.
Par exemple, la bibliothèque HOA~\cite{colafrancesco_bibliotheque_2013} permet l'encodage et le décodage ambisonique ou binaural. Elle fournit en même temps des outils permettant de modifier et travailer la position des sons dans l'espace, et qui peuvent donc être utilisés à fin d'écriture.

Nous étudions aussi le cas de l'audio dans un cadre de systèmes interactifs : ce sont des systèmes soit très statiques comme dans un audio-guide, soit très dynamiques mais avec une boucle d'exécution simple.
Par exemple, il est possible d'apposer des couches audio dans un espace réel à l'aide de cartes pour faire des parcours\cite{lemordant_augmented_2010}. 


\subsubsection{Composition et écriture}
Une présentation de l'état actuel de l'écriture spatiale en musique est donnée dans~\cite{fober_les_2015}. Notamment, la question de la notation dans le cadre de partitions impliquant des éléments spatiaux est abordé. % CITER PARTITIONS GRAPHIQUES.
Ces partitions peuvent être spatiales uniquement dans leur représentation, mais peuvent aussi indiquer des manières d'interpréter dans l'espace, notamment à l'aide de symboles spécialisés~\cite{ellberger_spatialization_2014}. Une taxinomie des possibilités de création dans l'espace en musiques électro-acoustiques est présentée par Bertrand Merlier dans~\cite{merlier_vocabulaire_2006}. Elle est étendue dans l'ouvrage \textit{Vocabulaire de l'espace en musique électro-acoustique\cite{merlier_vocabulaire_2006_book}}.

Des outils logiciels existent pour ces partitions -- ils sont souvent spécialisés. Par exemple, la bibliothèque \ac{ENP}\cite{kuuskankare_expressive_2006} permet de concevoir des partitions graphiques à l'aide d'un éditeur lui aussi graphique et d'un langage basé sur LISP.

Une des problématiques pour la représentation de l'écriture musicale est celle du geste, et de son lien avec la partition : comment notamment annoter le geste du musicien avec précision ?

Il existe un second problème de dissociation : celui entre la forme graphique et le résultat. Ainsi, des outils tels que % SOURCE
permettent de  travailler avec des trajectoires, mais uniquement pour des objets audio.

- Partitions graphiques
- IanniX

\subsubsection{Composition dans des environnements en 3D}
Une approche particulière est celle de l'immersion dans un monde virtuel en trois-dimensions pour la composition de musique. Cette approche est présentée par Michael Wozniewski dans~\cite{wozniewski_framework_2006}. Cela permet de définir et visualiser la propagation potentielle des ondes sonores dans un espace donné. De plus, en utilisant des méthodes de tracking de mouvement et des casques de réalité virtuelle, il est possible d'ouvrir de nombreuses possibilités d'interaction manuelle avec des objets virtuels. L'implémentation est faite en PureData.

D'autres possibilités d'utilisation d'environnement 3d à des fins d'écriture musicale sont suivies par Florent Berthaut avec le DRILE\cite{berthaut_drile:_2010}. Notamment, l'interaction spatiale est utilisée pour manipuler une hiérarchie plus aisément que via des interactions WIMP traditionnelles.

Dans les deux cas, ce sont des approches orientées vers la manipulation d'objets.

\subsubsection{Acoustique virtuelle}
Il existe plusieurs méthodes pour simuler un environnement virtuel (comme par exemple dans un jeu vidéo) sur le plan acoustique, en tenant compte de la géométrie dans laquelle peut se trouver le joueur (les approches les plus basiques utilisent simplement une atténuation linéaire entre la position du microhpone virtuel, souvent appelé \textit{listener}, et les sources sonores qu'il peut y avoir).

Cependant, il existe peu d'approches permettant d'être créatif par rapport aux possibilités qui sont offertes par l'outil informatique : notamment, les sources sont souvent ponctuelles alors que rien n'empêche de réfléchir à des sources planes ou volumétriques. 

Le rendu se fait généralement à l'aide de méthodes de lancer de rayon, inspirées des procédés utilisés en images de synthèse \cite{funkhouser_beam_1998,tsingos_fast_1998}.
Un procédé courant est maintenant d'utiliser les \ac{GPU} pour effectuer des calculs lourds mais facilement parallélisables\cite{rodriguez_performance_2014,cheng_design_2014,taylor_guided_2012}. Il est aussi possible de précalculer une fonction de convolution correspondant aux environnements virtuels que l'on veut parcourir, cependant cela demande des ressources en mémoire très importantes\cite{raghuvanshi_parametric_2014}.

\subsubsection{Restitution spatiale}
Il existe de nombreuses méthodes pour effectuer un rendu de son spatial. Certaines s'intéressent au rendu d'une "scène" spatiale donnée le mieux possible à l'aide d'un certain nombre de hauts-parleurs, tandis que d'autres visent la simulation d'une acoustique virtuelle comme nous l'avons vu précédemment. Comme pour HOA, les outils de création sont souvent fortement associés au système de rendu. La bilbiothèque 3Dj\cite{perez-lopez_3dj_2015} pour Supercollider vise à séparer cela à l'aide d'une architecture modulaire séparant le mapping d'interface utilisateur, la gestion de scène spatiale, les comportements, les modèles physiques, l'export et le rendu via méthodes ambisoniques, HRTF\cite{noisternig_3d_2003}, VBAP, DBAP et WFS.

La communication entre la gestion de la scène et le rendu se fait via le format standardisé SpatDIF\cite{peters_spatial_2013} par OSC.

\cite{bates_comparative_2007},\cite{tang_assistive_2014},\cite{sasamoto_controlling_2013},\cite{delerue_spatialisation_2004}

- Utilisation de données spatiales pour la création artistique \cite{naveda_topos_2014}

- Les objets qui étaient auparavant simplement connectés ont maintenant des infos de positions (téléphone, etc.) avec lesquelles il est possible de travailler (\cite{beal_spacetime_2015})
\subsection{Pratique liée à l'interaction}
\subsubsection{Jeu vidéo}
\cite{le_prado_ecriture_2013}
\cite{salazar_modelisation_2004}

\subsubsection{Systèmes interactifs}
\cite{chalon_realite_2004}
\cite{jankowski_advances_2015}
\cite{shen_blowbrush:_2014}
\cite{gustafson_imaginary_2010}
\subsubsection{Muséographie}
\cite{adhitya_composing_2012}
\cite{michael_comparative_2010}
\cite{azough_modeet_2014}
\cite{kidd_multi-touch_2011}
\subsubsection{Robots}
\cite{lee_virtual_2014} % NOTE : voir dans les références 1er paragraphe. 

\section{Manipulation et écriture}

\subsection{Extraction de données}
\cite{li_aesthetic_2009} (cf. slide 26)
\subsection{Trajectoires et animation}
\cite{santosa_direct_2013}
\cite{kazi_kitty:_2014,scott_physink:_2013}

\cite{garcia_jeremie_processing_2015,garcia_towards_2015}
\cite{wakefield_cosm:_2011}
\cite{wagner_introducing_2014}
\cite{melchior_authoring_2005} 
\cite{bresson_spatial_2012}
\cite{wozniewski_spatosc:_2012}
\cite{favory_trajectoires:_2015}
\cite{casas_4d_2013}


\subsubsection{Manipulation virtuelle}
\cite{jacob_design_2014} % NOTE: état de l'art intéressant au début.
\cite{cohen_interface_1999}
\subsubsection{Manipulation réelle}
\cite{grossman_interface_2003}


\subsection{Autres approches}
\cite{andriamarozakaniaina_du_2012} : Génération d'espace à partir de texte !
\cite{van_nort_lom_2006}
 - Zones ? logiciels de 3d paramétrique, etc.

\section{Modélisation}
\cite{porter_handbook_2008}
\cite{grenon_formal_2003}
\cite{zhang_timed_2014}
\cite{benford_spatial_1993}

\cite{matlage_every_2011}
\subsection{Modèles de données}
- OWL : \cite{mefteh_approche_2013}

- Adaptés à l'audio
. SpatDIF et compagnie \cite{peters_spatial_2013}\cite{kendall_towards_2008}
. \cite{kondoz_object-based_2014}
- Généralistes
\cite{hudak_arrows_2003}

Approches langage : (pour reconnaissance : \cite{spranger_recruitment_2011}\cite{spranger_emergent_2012}, )

- Modèles qualitatifs : \cite{chen_survey_2015}, \cite{bhatt_geospatial_2014}, \cite{schlieder_qualitative_1996,dorr_qualitative_2014}. Spatio-temporel : \cite{hazarika_qualitative_2012}. \cite{clementini_global_1997}.

- Voxels : \cite{kaufman_volume_1993}
- Quadtrees : \cite{eppstein_skip_2008}
\subsection{Problèmes géométriques et calculatoires}

\section{Rendu et visualisation}
\cite{hortner_spaxels_2012}
\subsection{Sur écran}
\subsection{Autre}
\printbibliography
\end{document}