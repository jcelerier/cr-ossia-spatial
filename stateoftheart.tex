\documentclass[french,12pt]{article}
\usepackage[T1]{fontenc}
\usepackage[utf8]{inputenc}
\usepackage{kpfonts}
\usepackage{babel}
\usepackage{microtype}
\usepackage{graphicx}
\usepackage{geometry}
\usepackage{hyperref}
\usepackage[backend=biber]{biblatex}
\bibliography{document.bib} % or
\geometry{hmargin=2.5cm,vmargin=1.5cm}

\title{État de l'art - spatial}
\begin{document}
\maketitle

\section{Pratiques}
\subsection{Pratique appliquée à l'audio}
- est utile pour l'audio en réalité augmentée\cite{lemordant_augmented_2010}
\subsubsection{Composition et écriture}
\cite{colafrancesco_bibliotheque_2013}
\subsubsection{Positionnement spatial}
\subsubsection{Restitution spatiale}
- Rendu de son dans des espaces virtuels \cite{funkhouser_beam_1998,cheng_design_2014,tsingos_fast_1998,taylor_guided_2012,raghuvanshi_parametric_2014,rodriguez_performance_2014}
Outils pour la création de tels espaces (funkhouser), 
- Méthodes : WFS, Amibisonie... \cite{lim_3d_2015}
\cite{perez-lopez_3dj_2015}
\cite{noisternig_3d_2003}\cite{bates_comparative_2007},\cite{tang_assistive_2014},\cite{sasamoto_controlling_2013},\cite{delerue_spatialisation_2004}

- Utilisation de données spatiales pour la création artistique \cite{naveda_topos_2014}

\subsection{Pratique liée à l'interaction}
\subsubsection{Jeu vidéo}
\subsubsection{Systèmes interactifs}
\subsubsection{Muséographie}

\section{Manipulation et écriture}
\subsection{Trajectoires et animation}
\cite{garcia_jeremie_processing_2015,garcia_towards_2015}
\cite{wakefield_cosm:_2011}
\cite{wagner_introducing_2014}
\cite{melchior_authoring_2005} 
\cite{bresson_spatial_2012}
\cite{wozniewski_spatosc:_2012}
\cite{favory_trajectoires:_2015}
\subsubsection{Manipulation virtuelle}
\subsubsection{Manipulation réelle}

\subsection{Zones et espaces}

\section{Modélisation}
\subsection{Modèles de données}
- Adaptés à l'audio
. SpatDIF et compagnie \cite{peters_spatial_2013}\cite{kendall_towards_2008}
. \cite{kondoz_object-based_2014}
- Généralistes

\subsection{Problèmes géométriques et calculatoires}

\section{Rendu et visualisation}
\subsection{Sur écran}
\subsection{Autre}
\printbibliography
\end{document}