\documentclass[french,12pt]{article}
\usepackage[T1]{fontenc}
\usepackage[utf8]{inputenc}
\usepackage{kpfonts}
\usepackage{babel}
\usepackage{microtype}
\usepackage{graphicx}
\usepackage{geometry}
\usepackage{hyperref}
\usepackage{acronym}
\usepackage[backend=biber]{biblatex}
\bibliography{document.bib} % or
\geometry{hmargin=2.5cm,vmargin=1.5cm}

\title{État de l'art - spatial}

\acrodef{ENP}{Expressive Notation Package}
\acrodef{GPU}{Graphics Processing Unit}
\acrodef{DBAP}{Distance-Based Amplitude Panning}
\acrodef{VBAP}{Vector Base Amplitude Panning}
\acrodef{WFS}{Wave-Field Synthesis}
\acrodef{GIS}{Geographic Information Service}
\begin{document}
\maketitle

\section{Introduction}
Ce document vise à présenter un état de l'art des partiques impliquant la manipulation de 
données spatiales dans un cadre artistique, interactif et muséographique.

Nous présentons notamment les besoins des concepteurs, ainsi que les diverses solutions qui 
peuvent y être apportées, que ce soit par la conception de logiciels dédiés ou bien 
par des modélisations formelles des notions d'espace.
 
 
 - Délimitation du périmètre
 - Définitions de l'espace dans les différentes pratiques ?
  * Lien très fort avec la VR et les mondes virtuels - jeux vidéos.
  * Audio spatialisé
  * Espace de données et manipulation de données multi-dimensionnelles
  * Muséographie
  * Big data ?
  * Reconaissance d'images 
  * Conception de spectacles / chorégraphie / etc

Beaucoup d'overlap.
  
Le problème qui nous intéresse principalement est celui de l'écriture. Cela implique des recherches sur les applications faisant usage d'écriture spatiale, des logiciels interagissant avec des données spatiales, des modèles aptes à représenter de telles données et des paradigmes d'interfaces utilisateur permettant de les modifier et de les retravailler. 



\section{Pratiques}
Art : \cite{lodi_spatial_2014}. L'espace dans une dimension collective. Exemple : Steve Eley, Licorne Rose et Invisible. Structure virtuelle apposée dans le monde réel et visible seulement à l'aide d'une application dédiée. Progression de la réalité virtuelle : la tendance est dans l'augmentation de la réalité par l'ajout d'objets qui extraient ou rendent des données (Manovich, 2005). "Spatial Art overlays and unites several spaces into one, making artistic use of time, movement and data or information in a space defined by growth in technological interaction, i.e., a data space."

Nous étudierons d'abord le cas de l'audio et de la spatialisation en audio (mettre historique).

\subsection{Pratique appliquée à l'audio}
Utilisation dans la musique, ainsi que dans des applications de réalité augmentée.

Problème principal dans cadre musical : couplage fort entre l'écriture et le moyen de restitution. Par exemple il est souvent nécessaire d'ajuster les paramètres de spatialisation quand on change l'endroit ou se déroule une représentation. %TODO trouver source.
Par exemple, la bibliothèque HOA~\cite{colafrancesco_bibliotheque_2013} permet l'encodage et le décodage ambisonique ou binaural. Elle fournit en même temps des outils permettant de modifier et travailer la position des sons dans l'espace, et qui peuvent donc être utilisés à fin d'écriture.

Nous étudions aussi le cas de l'audio dans un cadre de systèmes interactifs : ce sont des systèmes soit très statiques comme dans un audio-guide, soit très dynamiques mais avec une boucle d'exécution simple.
Par exemple, il est possible d'apposer des couches audio dans un espace réel à l'aide de cartes pour faire des parcours\cite{lemordant_augmented_2010}. 


\subsubsection{Composition et écriture}
Une présentation de l'état actuel de l'écriture spatiale en musique est donnée dans~\cite{fober_les_2015}. Notamment, la question de la notation dans le cadre de partitions impliquant des éléments spatiaux est abordé. % CITER PARTITIONS GRAPHIQUES.
Ces partitions peuvent être spatiales uniquement dans leur représentation, mais peuvent aussi indiquer des manières d'interpréter dans l'espace, notamment à l'aide de symboles spécialisés~\cite{ellberger_spatialization_2014}. Une taxinomie des possibilités de création dans l'espace en musiques électro-acoustiques est présentée par Bertrand Merlier dans~\cite{merlier_vocabulaire_2006}. Elle est étendue dans l'ouvrage \textit{Vocabulaire de l'espace en musique électro-acoustique\cite{merlier_vocabulaire_2006_book}}.

Des outils logiciels existent pour ces partitions -- ils sont souvent spécialisés. Par exemple, la bibliothèque \ac{ENP}\cite{kuuskankare_expressive_2006} permet de concevoir des partitions graphiques à l'aide d'un éditeur lui aussi graphique et d'un langage basé sur LISP.

Une des problématiques actuelles pour la représentation de l'écriture musicale est celle du geste, et de son lien avec la partition : comment notamment annoter le geste du musicien avec précision ? Et, inversement, comment à partir d'un geste créer un son correspondant ? Ces questions sont abordées dans la description de Soundstudio 4D\cite{sheridan_soundstudio_2004}, dans le cadre d'un système de conception de trajectoires pour spatialisation à l'aide d'interactions en trois dimensions.

Une autre question est l'association entre l'aspect graphique et le résultat. Ainsi, des outils tels que HoloEdit et HoloSpat permettent de travailler avec des trajectoires, mais sont extrêmement spécialisés pour des objets audio. C'est notamment du à la nécessité de composer en ayant conscience à chaque instant des fortes contraintes techniques du moyen de restitution de l'oeuvre. Il serait intéressant d'utiliser ces trajectoires pour contrôler non pas des sources sonores mais des éléments dans des espaces de paramètres quelconques.

Le logiciel IanniX\cite{jacquemin_iannix_2012} dispose aussi de nombreuses possibilités d'écriture spatiale : les partitions sont des ensembles d'éléments graphiques définis paramétriquement ou bien à l'aide d'un langage de programmation dédié, que des curseurs vont parcourir. L'information de position de chaque curseur est envoyée en OSC, ce qui permet l'intégration à d'autres logiciels.

Une méthode d'écriture de la spatialisation par contraintes est proposée par Olivier Delerue avec le système MusicScpace\cite{delerue_spatialisation_2004}. Cela permet une approche déclarative à l'écriture de partition, en spécifiant des contraintes telles que "deux objets ne doivent jamais être à plus de deux mètres l'un de l'autre" ou bien "l'angle entre deux objets et l'auditeur doit être supérieur à 90 degrés". Les objets peuvent être notamment des sources sonores. Une édition graphique de ces contraintes est proposée, et elles sont représentées en termes de cercles et de segments reliant les objets qu'elles contraignent.

Des données spatiales peuvent aussi être utilisées directement pour créer des mappings sonores. C'est le cas notamment de la bibliothèque Topos\cite{naveda_topos_2014}, qui permet de capter le mouvement de danseurs et d'en extraire des informations pouvant être utiles pour la conception de pièces de musique interactives. Une fois que le mouvement du danseur est capturé via un périphérique externe, il devient possible d'extraire des informations telles que le volume occupé par le danseur, sa vitesse, ou bien diverses mesures relatives à l'évolution de deux ou quatres points dans le temps, comme l'instabilité ou les collisions entre différentes parties du corps. Ces données peuvent ensuite être réutilisées dans Pure Data pour de la génération de musique.
\subsubsection{Interaction spatiale pour création de son}
,\cite{sasamoto_controlling_2013},

\subsubsection{Composition dans des environnements en 3D}
Une approche particulière est celle de l'immersion dans un monde virtuel en trois-dimensions pour la composition de musique. Cette approche est présentée par Michael Wozniewski dans~\cite{wozniewski_framework_2006}. Cela permet de définir et visualiser la propagation potentielle des ondes sonores dans un espace donné. De plus, en utilisant des méthodes de tracking de mouvement et des casques de réalité virtuelle, il est possible d'ouvrir de nombreuses possibilités d'interaction manuelle avec des objets virtuels. L'implémentation est faite en PureData.

D'autres possibilités d'utilisation d'environnement 3d à des fins d'écriture musicale sont suivies par Florent Berthaut avec le DRILE\cite{berthaut_drile:_2010}. Notamment, l'interaction spatiale est utilisée pour manipuler une hiérarchie plus aisément que via des interactions WIMP traditionnelles.

Dans les deux cas, ce sont des approches orientées vers la manipulation d'objets.

\subsubsection{Acoustique virtuelle}
Il existe plusieurs méthodes pour simuler un environnement virtuel (comme par exemple dans un jeu vidéo) sur le plan acoustique, en tenant compte de la géométrie dans laquelle peut se trouver le joueur.

Les approches les plus simples utilisent simplement une atténuation linéaire entre la position du microphone virtuel, souvent appelé \textit{listener}, et les sources sonores qu'il peut y avoir. Différents procédéss, comme ceux proposés par la technologie EAX\cite{funkhouser_survey_2003}, permettent d'ajouter une forme d'occlusion basique, ainsi que des effets de réverbération statique dans des endroits donnés pour enrichir l'impression d'espace. Des outils plus récents tels que FMOD et wWise s'intègrent notamment avec les moteurs de jeux pour pouvoir tirer parti directement des objets positionnés dans l'espace du moteur de jeu.

Cependant, il existe peu d'approches permettant d'être créatif par rapport aux possibilités qui sont offertes par l'outil informatique : notamment, les sources sont souvent ponctuelles alors que rien n'empêche de réfléchir à des sources planes ou volumétriques. 

Le rendu se fait généralement à l'aide de méthodes de lancer de rayon, inspirées des procédés utilisés en images de synthèse \cite{funkhouser_beam_1998,tsingos_fast_1998}.
Un procédé courant est maintenant d'utiliser les \ac{GPU} pour effectuer des calculs lourds mais facilement parallélisables\cite{rodriguez_performance_2014,cheng_design_2014,taylor_guided_2012}. Il est aussi possible de précalculer une fonction de convolution correspondant aux environnements virtuels que l'on veut parcourir, cependant cela demande des ressources en mémoire très importantes\cite{raghuvanshi_parametric_2014}.

\subsubsection{Restitution spatiale}
Il existe de nombreuses méthodes pour effectuer un rendu de son spatial. Certaines s'intéressent au rendu d'une "scène" spatiale donnée le mieux possible à l'aide d'un certain nombre de hauts-parleurs, tandis que d'autres visent la simulation d'une acoustique virtuelle comme nous l'avons vu précédemment. Comme pour HOA, les outils de création sont souvent fortement associés au système de rendu. La bilbiothèque 3Dj\cite{perez-lopez_3dj_2015} pour Supercollider vise à séparer cela à l'aide d'une architecture modulaire séparant le mapping d'interface utilisateur, la gestion de scène spatiale, les comportements, les modèles physiques, l'export et le rendu via méthodes ambisoniques, HRTF\cite{noisternig_3d_2003}, \ac{VBAP}, \ac{DBAP} et \ac{WFS}. Une autre proposition d'architecture modulaire, par strates, est proposée par Nils Peters dans~\cite{peters_stratified_2009}.

La communication entre la gestion de la scène et le rendu se fait via le format standardisé SpatDIF\cite{peters_spatial_2013} par OSC. % TODO décrire plus loin

Les différentes méthodes de spatialisation citées sont comparées via la capacité des auditeurs à situer la présence d'une source sonore dans l'espace dans~\cite{bates_comparative_2007}, et la précision de leur localisation.

L'intérêt de telles restitutions n'est pas seulement artistique : l'audio spatialisé peut être utilisé pour combler les déficiences visuelles, par exemple via un processus de sonification\cite{tang_assistive_2014}. Cela implique notamment une phase de reconnaissance d'objets, puis une phase de mapping d'un espace visuel défini hiérarchiquement vers un espace de sons qui permettra à la personne malvoyante de reconnaître des objets inertes via le son qui leur sera associé.

\subsection{Pratique liée à l'interaction}

% NOTE : faire partie qui différencie création via objets, création de mappings création de trajectoires et qui compare les différentes approches que l'on a vu.
\subsubsection{Jeu vidéo}
\cite{le_prado_ecriture_2013}
\cite{salazar_modelisation_2004}

\subsubsection{Systèmes interactifs}

- Les objets qui étaient auparavant simplement connectés ont maintenant des infos de positions (téléphone, etc.) avec lesquelles il est possible de travailler (\cite{beal_spacetime_2015})

- Modèle d'interaction homme-machine pour la réalité augmentée, avec de nombreuses définitions : \cite{chalon_realite_2004}

- État de l'art complet sur l'interaction avec environnements 3D.
\cite{jankowski_advances_2015}

- Production de dessin à l'aide du souffle
\cite{shen_blowbrush:_2014}

- Interfaces imaginaires : il existe un monde en 3D superposé au monde en 2D existant.
\cite{gustafson_imaginary_2010}

\subsubsection{Muséographie}
- Création de partition à partir d'information de couleur d'une image en 2D. Zones et progressions.
\cite{adhitya_composing_2012}

- Comparaison d'installations interactives dans les musées : ce n'est pas parce qu'il y a un grand nombre de degrés de liberté en entrée ni que le contenu est en 3 dimensions que des enfants prennent du plaisir et ont envie de revenir participer à l'installation.
\cite{michael_comparative_2010}

- Utilisation du son pour enrichir les visites des musées, en permettant la création de paysages sonores.
\cite{azough_modeet_2014}

- Place de l'interaction multitouch dans musées.
\cite{kidd_multi-touch_2011}

- Conception de prévisualisations
\cite{jung_storyboarding_2010}

\subsubsection{Robots}
- Utilisation de robots lors d'une chorégraphie. Positionnements des jointures organisées hiérarchiquement dans l'espace. Souvent, nécessité de rendu dans un environnement virtuel. 
\cite{lee_visualization_2013}
Puis, outil de conception complet de robot dans~\cite{lee_virtual_2014} à l'aide de moteurs virtuels modélisés en 3D et d'un calculateur de couple. Estimation de vitesse du robot à l'aide de motion capture sur 25 degrés de liberté appliquée à un robot réel.

- Génération de pas de dance pour un robot à partir de musique.
\cite{seo_autonomous_2013}

\section{Manipulation et écriture}

\subsection{Extraction de données}
- Présente 26 features qui peuvent être extraites de tableaux et qui permettent de mesurer les qualités esthétiques, puis de gérer des classifications. Basé sur questionnaire sur 42 participants et 100 tableaux. \cite{li_aesthetic_2009} (cf. slide 26).
Example : clarté moyenne logarithmique, forme des segments.

\subsection{Trajectoires et animation}
- Rhonda\footnote{http://rhondaforever.com/}, utilisé dans Sonic Wire Sculptor et Ink Space (cf. openframeworks). Détaillé dans~\cite{rasmuson_flying_2013}.

- En animation : keyframes, et rotoscoping. Problématique principale : lier l'espace et le temps. Dans les outils existants, le contrôle du flot du temps est séparé de la création / peinture / etc. Utiliatio nde crayon pour contrôler le temps via les trajectoires.
\cite{santosa_direct_2013}

- Autre possibilité : textures kinétiques : les textures contiennent une ifnormation de déplacement. Plusieurs formes de textures : textures émettrices qui génèrent des objets, et textures oscillantes. Utilisation d'un graphe relationnel pour définir les interactions qu'il peut y avoir entre les différentes entités.
\cite{kazi_kitty:_2014}

- Description de comportements physiques via dessin\cite{scott_physink:_2013}. Fonctionne en 2D.

\cite{garcia_jeremie_processing_2015,garcia_towards_2015} % TODO bof : plutôt pour gestion du son

- Modèle d'espace en 3d dans Max/MSP. En plus de lieux et de trajectoires, il est possible d'écrire l'interaction dans une certaine mesure, ainsi que la communication entre différents agents. Représentation de champs pouvant varier dans le temps et pouvant être sonifiés.
\cite{wakefield_cosm:_2011}

- Zirkonium : étideur pour spatialisation. Utilisation de courbes de Bézier. 
\cite{wagner_introducing_2014}

- Système de spatialisation \ac{WFS} par réalité augmentée. Contient une liste des interactions possibles : disposition spatiale des sources, effets audio, simulation de pièce... Discussion sur la visualisation de sources sonores, 
\cite{melchior_authoring_2005} 

- Utilisation de OpenMusic pour composition visuelle de la spatialisation. Scenes spatiales sont représentes par matrices source - paramètres de spat. Trajectoires en coordonnées cartésiennes, et time-tag. Utilisation des données spatiales directement au moment de la synthèse.
\cite{bresson_spatial_2012}

- bibliothèque de spatialisation en C++, permettant la traduction entre différents formats de données spatiales audio.
\cite{wozniewski_spatosc:_2012}

- conception de trajectoires à l'aide de dispositifs mobiles multitouch. Étude avec plusieurs compositeurs, qui ont chacun utilisé le logiciel à leur manière et discuté cette utilisation.
\cite{favory_trajectoires:_2015}

- Graphe de motion : représentation de l'évolution d'une animation au cours du temps en temps réel. Méthode utilisée : fusion de maillages, permettant des transitions entre différentes animations suite à motion capture. % TODO bof....
\cite{casas_4d_2013}


\subsubsection{Manipulation virtuelle}
- État de l'art sur l'édition de courbes 3D en séparant en deux axes : le dispositif physique (entrée - sortie) et le langage d'interaction (création / édition de courbe, et manipulation de la caméra).
\cite{jacob_design_2014} % NOTE: état de l'art intéressant au début.

- Manipulation de courbes 3D en s'aidant des ombres.
\cite{cohen_interface_1999}

\subsubsection{Manipulation réelle}
- Manipulation tangible de courbes à l'aide d'un périphérique spécifique, ShapeTape, qui est déformable et permet d'obtenir informatiquement sa torsion en 32 points pour reconstruire la courbe de manière virtuelle en 3D : 
\cite{grossman_interface_2003}

\subsection{Autres approches}
- Génération d'espace à partir de description textuelle.
\cite{andriamarozakaniaina_du_2012} 

- Conception de mappings en trois dimensions dans Max/MSP. Mappings de régions de l'espace via interpolation : le mapping a lieu entre N dimensions de contrôle et M dimensions sonores.
\cite{van_nort_lom_2006}


 - Zones ? logiciels de 3d paramétrique, etc. % TODO parler des approches standard par keyframes, etc. au début

\section{Modélisation}
- Généralement : représentations quantitatives (x, y, z) ou qualitatives (via logique).
\cite{porter_handbook_2008}

- Construction d'une ontologie du raisonnement spatio-temporel basé sur la notion de réalité. Séparation entre évènements \textit{continuants} et \textit{occurents}. Ces composants de l'ontologie générale présentée servent de point de vue sur la réalité et peut ou non interagir avec les autres. La persistence des continuants se fait par l'existence de parties temporelles successives; le temps est une dimension de spécification des occurents mais pas des continuants.
\cite{grenon_formal_2003}

- possibilités probabilistes

- Modèle d'automates spatio-temporels adaptés à la gestion de systèmes de transports % TODO bof....
\cite{zhang_timed_2014}

- Modèle d'espace pour l'interaction en réalité virtuelle. Apporte les notions de moyen, d'aura, d'awareness..
\cite{benford_spatial_1993}

- Modélisation fonctionnelle des animations via un DSL dédié en haskell. Présupposé : toute animation a un début, un milieu et une fin. Combinaison d'animations.
\cite{matlage_every_2011}

\subsection{Modèles de données}
Ontologie pour le web. Application aux \ac{GIS}.
- OWL : \cite{mefteh_approche_2013}.

- X3DOM :langage déclaratif intégré au DOM HTML/CSS\cite{jankowski_declarative_2013}. Par opposition à WebGL qui est impératif.

- Adaptés à l'audio
. SpatDIF et compagnie \cite{peters_spatial_2013}\cite{kendall_towards_2008}
. \cite{kondoz_object-based_2014}

- Généralistes
- Programmation fonctionnelle réactive; là aussi, un DSL dédié en Haskell qui utilise les Arrow.
\cite{hudak_arrows_2003}

Approches langage : (pour reconnaissance : \cite{spranger_recruitment_2011}\cite{spranger_emergent_2012}, )

- Modèles qualitatifs : \cite{chen_survey_2015}, \cite{bhatt_geospatial_2014}, \cite{schlieder_qualitative_1996,dorr_qualitative_2014}. Spatio-temporel : \cite{hazarika_qualitative_2012}. \cite{clementini_global_1997}.

- Trajectoires : via équations paramétriques.

- Voxels : \cite{kaufman_volume_1993}

- Quadtrees : \cite{eppstein_skip_2008}
\subsection{Problèmes géométriques et calculatoires}

\section{Rendu et visualisation}
- Utilisation de drones portant des lumières pour faire des pixels dans l'espace.
\cite{hortner_spaxels_2012}
\subsection{Sur écran}
\subsection{Autre}
\printbibliography
\end{document}